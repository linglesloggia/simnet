%\documentclass[spanish, compress]{beamer}
% from lankton-theme demo tex file
\documentclass[spanish]{beamer}    % [serif, mathserif] obsolete, suppressed

\usepackage{amsmath, amsfonts, epsfig, xspace}
%\usepackage{algorithm,algorithmic}
\usepackage{pstricks,pst-node}
\usepackage{multimedia}
\usepackage[normal,tight,center]{subfigure}
\setlength{\subfigcapskip}{-.5em}
\usepackage{beamerthemesplit}

%\usetheme{lankton-keynote-vgb}
\useoutertheme{default}
\usecolortheme{seagull}

% spanish, accents
\usepackage[spanish]{babel}
\selectlanguage{spanish}
\usepackage[utf8]{inputenc}


\author[Pablo Belzarena, Víctor González Barbone]{Pablo Belzarena, Víctor González Barbone}

\title[SimNet: diseño\hspace{2em}\insertframenumber/\inserttotalframenumber]{SimNet: diseño}

\subtitle{A discrete event, system level\\ wireless network simulator and software framework}   %\vspace{0.1cm} 

\date{Diciembre 2023}     %leave out for today's date to be insterted

\institute[IIE]{Instituto de Ingeniería Eléctrica\\
   Facultad de Ingeniería\\
   Universidad de la República Uruguay  %\\
   %\vspace{0.2cm}
   %\includegraphics[height=1.0cm]{IIE\_FIng\_UR.jpeg}
   %\vfill
   }

\logo{\includegraphics[height=0.6cm]{iie_logo_1500.png}}

\begin{document}

% --------------------------------------------------Slide --

%\maketitle
\begin{frame}
  \titlepage
\end{frame}

% --------------------------------------------------Slide --

\begin{frame}
    \frametitle{Contenido}
    SimNet, presentación del diseño y estado de avance.
    \begin{itemize}
        \item Antecedentes y objetivos.
        \item Arquitectura.
        \item Escenario de simulación.
        \item Simulación por eventos.
        \item Cola de paquetes y manejo de transmisión.
        \item Integración con PyWiCh.
        \item Próximos objetivos.
    \end{itemize}
\end{frame}

% --------------------------------------------------Slide --

\begin{frame}
    \frametitle{Antecedentes y objetivos}
    Antecedentes:
    \begin{itemize}
        \item varios proyectos anteriores (tesis, fin de carrera).
        \item realizaciones de prototipo.
        \item difíciles de extender a diferentes modelos.
    \end{itemize}
    Objetivos:
    \begin{itemize}
        \item buen diseño de arquitectura, modularización, bajo acoplamiento, alta cohesión; un "software framework".
        \item eficiencia: estructuras de datos y su manejo.
        \item integrar desarrollos de proyectos anteriores.
        \item integrar e interactuar (datos de archivo) con simulador de canal PyWich: ejemplo y prueba de concepto de extensión.
        \item orientado a la extensión: funcionalidad, protocolos, algoritmos.
        \item mantener la evolución del software con estos objetivos.
    \end{itemize}
\end{frame}

% --------------------------------------------------Slide --

\begin{frame}
    \frametitle{Diagrama de paquetes}
    \begin{columns}
    \begin{column}{0.50\textwidth}
        Biblioteca libsimnet:
        \begin{itemize}
            \item núcleo del sistema.
            \item se pueden sobreescribir.
            \item solo desarrolladores SimNet.
        \end{itemize}
        Extensiones:
        \begin{itemize}
            \item para distintos modelos de simulación.
            \item agregan clases, sobreescriben las del núcleo.
            \item en SimNet, dentro del paquete \texttt{extensions}, o
            \item fuera de SimNet, ``out of tree''.
        \end{itemize}
    \end{column}
    \begin{column}{0.50\textwidth}
        \pgfimage[width=0.99\textwidth]{../diagrams/UMLpackages.png}
    \end{column}
    \end{columns}
\end{frame}

% --------------------------------------------------Slide --

\begin{frame}[plain]
    \frametitle{Diagrama de clases}
    \center \pgfimage[width=0.95\textwidth]{../diagrams/UML-libsimnet_clases.png}
\end{frame}

% --------------------------------------------------Slide --

\begin{frame}[plain]
    \frametitle{Diagrama de objetos}
    \center \pgfimage[width=0.95\textwidth]{../diagrams/UML-libsimnet_objects.png}
\end{frame}

% --------------------------------------------------Slide --

\begin{frame}
    \frametitle{Estructura de directorios}
    SimNet, directorio del proyecto. Contiene:
        \begin{itemize}
            \item \texttt{libsimnet} : el núcleo, modifican solo desarrolladores.
            \item \texttt{extensions/xxxx\_sim} : clases agregadas y sobreescritas de \texttt{libsimnet} para simulador\ Xxxx. Incluye setup de escenario y código de prueba.
            \item sandbox : códigos de prueba de algoritmos y estructuras.
            \item docs : documentación formato MD (markdown language), usados en Gitlab, Github, etc.
            \item html : documentación del código, con pydoctor. 
        \end{itemize}
    Un nuevo modelo de simulación puede implementarse:
        \begin{itemize}
            \item fuera de la estructura de directorios de SimNet, o
            \item dentro del directorio \texttt{extensions}.
        \end{itemize}
\end{frame}

% --------------------------------------------------Slide --

\begin{frame}
    \frametitle{Escenario de simulación}
        %\begin{itemize}
        %    \item Concentra en un módulo la definición del escenario de simulación, para poder probar fácilmente distintos escenarios
        Concentra en un módulo la definición del escenario de simulación, para poder probar fácilmente distintos escenarios.\\
        %\vspace{0.2cm}
        Clase \texttt{simsetup.Setup}, crea escenario en base a lista de items:\\
            \vspace{0.2cm}
            \texttt{[ class\_name, nr\_items, attach\_to, ls\_pars, ... ]} 
            \vspace{0.2cm}
            \begin{itemize}
                \item define objetos, cantidad y a qué entidad se enganchan.
                \item permite fijar parámetros para las distintas entidades.
                \item devuelve dos listas: \texttt{ls\_slices} (slices), y \texttt{ls\_trfgen} (generadores de tráfico).
            \end{itemize}
        En el módulo de usuario \texttt{qa\_simulator}, las funciones:
            \begin{itemize}
                \item \texttt{setup\_sim}: fija parámetros, crea objeto Setup.
                \item \texttt{run\_sim}: recibe objeto Setup, crea objetos Simulation y Statistics, corre la simulación. 
            \end{itemize}
%        \end{itemize}
\end{frame}

% --------------------------------------------------Slide --

\begin{frame}
    \frametitle{Simulación por eventos}
    Modos de simulación: alternativas, ventajas e inconvenientes: 
        \begin{itemize}
            \item simuladores genéricos, e.g. simpy, muchas funcionalidades.
            \item threads: muchos, overhead, coordinación entre sí.
            \item procesamiento secuencial con unidades de tiempo.
            \item eventos, en una cola de prioridad.
        \end{itemize}
    Clase de Python \texttt{queue.PriorityQueue}:
        \begin{itemize}
            \item inserción: \texttt{put(evento)}, ubica en lista ordenada.
            \item extracción: \texttt{get()} devuelve el evento más próximo.
            \item inserción en O(log(n)), extracción en O(1).
            \item el evento debe ser ordenable; en SimNet, una lista\\
            \texttt{[ time\_t, priority, id\_action]}
        \end{itemize}
    Eventos en SimNet:
        \begin{itemize}
            \item cada proceso periódico agenda su próximo evento.
            \item procesos: generar tráfico, transmitir, repartir recursos.
        \end{itemize}
\end{frame}

% --------------------------------------------------Slide --

\begin{frame}
    \frametitle{Eventos en SimNet}
    Tipos de evento, lista: \texttt{[id\_action, delay, priority]}
    \begin{itemize}
        \item \texttt{id\_action}: nombre de una función, ordenable, se busca en un diccionario \texttt{ \{id\_action : function\} }
        \item \texttt{delay}: siguiente evento en \texttt{time\_t + delay}.
        \item \texttt{priority}: para eventos que ocurren en un mismo momento.
        \item eventos: GenTraffic, GenerateTBs, TransmitTBs, AssignRes, ShowProgress, EndSimulation.
        \item delay en EndSimulation es la duración de la simulación.
    \end{itemize}
    Secuencia de simulación:
    \begin{itemize}
        \item al inicio, genera un evento de cada tipo.
        \item extrae un evento, sale siempre el más cercano.
        \item ejecuta la función correspondiente al evento extraído.
        \item la función devuelve un nuevo evento, cuyo delay puede variar.
        \item inserta el nuevo evento en la cola de prioridades.
    \end{itemize}
\end{frame}

% --------------------------------------------------Slide --

\begin{frame}
    \frametitle{PacketQueue, lista de paquetes}
    Formato de registro de un paquete:\\
        \texttt{[ id\_pkt, time\_rec, time\_sent, object | nr\_bits ]}
    \begin{itemize}
        \item los datos pueden ser un objeto, un string o un número de bits.
        \item una vez transmitidos, pueden conservarse o descartarse.
        \item la lista puede tener longitud máxima, al superarla se descartan los paquetes.
        \item transmisión por paquetes o por transport blocks.
        \item un transport block incluye uno o más paquetes, según recursos, canal, etc., al momento de transmitir.
        \item un transport block puede transmitirse con éxito o perderse.
        \item los paquetes perdidos se retienen y retransmiten con la mayor prioridad.
        \item contadores registran recibidos, enviados, retransmitidos, descartados, en número de paquetes y en bits.
    \end{itemize}
\end{frame}

% --------------------------------------------------Slide --

\begin{frame}[plain]
    \frametitle{PacketQueue, proceso de transmisión (I)}
    \center \pgfimage[width=0.97\textwidth]{../diagrams/pktqueue-transmit.png}
\end{frame}

% --------------------------------------------------Slide --

\begin{frame}
    \frametitle{PacketQueue, proceso de transmisión (II)}
\texttt{mk\_trblk(tb\_size)} devuelve un transport block (TB):
% \texttt{pkt\_qu.mk\_trblk(tb\_size)}; paquetes:
\begin{itemize}
    \item extrae del diccionario de retransmisión\\ \texttt{dc\_retrans = \{id\_packet : packet\};}
    \item extrae de la cola de paquetes recibidos;
    \item crea TB = \texttt{[tr\_blk\_id, pkt\_1, pkt\_2, ...] }
    \item agrega a diccionario de TBs pendientes (en aire):\\ \texttt{dc\_pend[tr\_blk\_id] = [pkt\_1, pkt\_2, ...] }
\end{itemize}
\vspace{0.2cm}
%Procesa envío, recibe identificador de TB y \texttt{"Lost"} o \texttt{"Sent"}:
\texttt{send\_tb(tr\_blk\_id, "Sent"|"Lost", t\_stamp)} procesa envío:
\begin{itemize}
    \item para cada paquete en \texttt{dc\_pend[tr\_blk\_id]} :
    \begin{itemize}
        \item IF "Sent":
        \begin{itemize}
            \item agrega timestamp al paquete; ajusta contadores. 
            \item opcionalmente lo mueve a la cola de paquetes enviados.
            \item si el paquete está en \texttt{dc\_retrans}, lo extrae.
        \end{itemize}
        \item IF "Lost":
            \begin{itemize}
                \item IF el paquete no está en \texttt{dc\_retrans}, lo agrega.
            \end{itemize}
    \end{itemize}
    \item extrae el TB de pendientes: \texttt{del dc\_pend[tr\_blk\_id] }
\end{itemize}

\end{frame}

% --------------------------------------------------Slide --

\begin{frame}[plain]
    \frametitle{PyWiCh, integración en SimNet}
    \center \pgfimage[width=0.95\textwidth]{../diagrams/UML-pywich_sim.png}
\end{frame}

% --------------------------------------------------Slide --

\begin{frame}
    \frametitle{Próximos objetivos}
    Objetivos inmediatos:
    \begin{itemize}
        \item en libsimnet, identificar clases y métodos a sobreescribir.
        \item crear un ejemplo simple como guía para la extensión de libsimnet.
        \item documentación, páginas de descripción, instalación, extensión.
        \item pruebas de rendimiento en distintos escenarios.
    \end{itemize}
    \vspace{0.5cm}
    Objetivos de próximo desarrollo:
    \begin{itemize}
        \item estudiar e integrar protocolos y algoritmos implementados en proyectos anteriores.
        \item adaptar interfaces de libsimnet para facilitar la extensión e integración de otras implementaciones de protocolos y algoritmos (mejora continua).
    \end{itemize}
\end{frame}

% --------------------------------------------------Slide --

\begin{frame}
    \frametitle{Final}
    \center \pgfimage[width=0.85\textwidth]{PdD-Piria.png}
    \vspace{0.5cm}
    \center Muchas gracias
    %\vspace{0.3cm}
    \center IIE en el Este
\end{frame}

% --------------------------------------------------Slide --

\end{document}
